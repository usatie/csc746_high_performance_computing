% Please take a few moments and try to compose an abstract for your homework writeup. It should contain these ideas: what was the problem being studied, what was the approach (what did you implement), what are the results.
% The abstract should describe the basic message of the paper, including: the problem, why your solution should be of interest, some notion that your solution is effective, and a teaser about how it has been evaluated. Cover all of this using between 75 and 150 words. Thus, the abstract is the hardest part to write. Sometimes I try to write it first, but the final version is usually composed of items drawn from the introduction, and then condensed, as the last step of writing the paper.

% describes the focus of the study, the approach, and the primary findings/results (3 or 4 sentences total). Writing tip: it's often the case that the Abstract and Introduction are the last items written in a technical paper, once you know the outcome of the performance study.

This study investigates the implementation of distributed-memory stencil operations using the Message Passing Interface (MPI)\cite{mpi_spec} to enhance computational performance through parallel execution across multiple CPU nodes. The objective was to design and evaluate a distributed Sobel filter for edge detection, employing three decomposition strategies: Row-slab, Column-slab, and Tiled. Custom MPI datatypes were utilized to optimize data exchange, particularly for non-contiguous memory regions, while ensuring scalable parallelization.

Experimental results show that all decomposition strategies improved runtime performance with increasing concurrency. The tiled decomposition strategy demonstrated superior scalability due to its even workload distribution. Additionally, the runtime for scattering and gathering was consistent across methods and concurrency levels, benefiting from the use of subarray datatypes to minimize inter-node communication. These findings underscore the importance of workload balance and communication efficiency in optimizing distributed-memory stencil operations.
