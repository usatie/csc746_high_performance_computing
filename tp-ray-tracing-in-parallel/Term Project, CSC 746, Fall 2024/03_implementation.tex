\section{Implementation}
\label{sec:implementation}
% Put an introductory paragraph here that gives the reader an overview of what's coming. If there are multiple subsections, say in a few words or a sentence something about each subsection.

% include a separate subsection for each of the different implementations. Briefly describe your implementation, and include the use of compact pseudocode as necessary. The focus here should be on conciseness and clarity. Be sure to describe the strategy you used in parallelizing your code. Your example Listings should clearly indicate the OpenMP pragmas you used. 
In this study, we implemented two versions of a ray tracing system: a serial implementation (Serial RT) and a parallel implementation using OpenMP (OpenMP RT). These implementations were designed to explore the performance differences and challenges in parallelizing ray tracing on multi-core CPUs.

\subsection{Ray tracing foundations}
\label{subsec:ray-tracing-foundations}
The objective of this section is to establish the core principles and computational steps underlying the ray tracing process. By defining methods for rendering an image, generating rays, determining ray colors through recursive evaluation, and testing for intersections, we set the groundwork for producing visually accurate, physically plausible images.

This section introduces the essential components of a ray tracing pipeline, from pixel-by-pixel rendering to ray generation, color computation, and intersection handling. The details, supported by pseudo-code and listings, show how rays are cast, how intersections and scattering are computed, and how final colors are accumulated.

\subsubsection{Rendering}
This subsection outlines the process of computing the final image by iterating over each pixel, generating multiple rays per pixel, computing their colors, and averaging the results, as shown in Algorithm \ref{alg:rendering}.

\begin{algorithm}[htbp]
\KwData{image\_height, image\_width, samples\_per\_pixel, max\_recursion\_depth, world}
initialize image\;
\For{$y = 0$ \KwTo $image\_height$}{
    \For{$x = 0$ \KwTo $image\_width$}{
        \For{$i = 0$ \KwTo $samples\_per\_pixel$}{
            $r \gets get\_ray(x, y)$\;
            $image[x, y] \gets image[x, y] + ray\_color(r, max\_recursion\_depth, world)$\;
        }
    }
}
write image\;
\caption{\textbf{Render Image.} Loop structure for rendering an image by averaging multiple samples per pixel.}
\label{alg:rendering}
\end{algorithm}

\FloatBarrier
\subsubsection{Generating Ray}
This subsection details how a ray is constructed from the camera to a specific pixel, incorporating randomness for anti-aliasing and depth-of-field effects, as shown in Listing \ref{listing:get-ray}.

\begin{lstlisting}[caption={\textbf{Generating a ray from the camera to a screen pixel}},label={listing:get-ray}, name=get-ray, float=htbp, style=mystyle,language=C++]
ray get_ray(int i, int j) const {
    auto offset = sample_square();
    auto pixel_sample = pixel00_loc + (i + offset.x()) * pixel_delta_u + (j + offset.y()) * pixel_delta_v;
    auto ray_origin = (defocus_angle <= 0) ? center : defocus_disk_sample();
    auto ray_direction = pixel_sample - ray_origin;
    return ray(ray_origin, ray_direction);
}
\end{lstlisting}

\FloatBarrier
\subsubsection{Computing a ray's color}
This subsection explains the recursive computation of a ray’s color as it interacts with objects, possibly scattering to produce reflections and other effects, or returning a background color if no objects are hit, as shown in Listing \ref{listing:ray-color}.

\begin{lstlisting}[caption={\textbf{Computing a ray's color}},label={listing:ray-color}, name=ray-color, float=htbp, style=mystyle,language=C++]
color ray_color(const ray &r, int depth, const hittable &world) {
    if (depth <= 0) return color(0, 0, 0);
    hit_record rec;
    if (world.hit(r, interval(0.001, infinity), rec)) {
      ray scattered;
      color attenuation;
      if (rec.mat->scatter(r, rec, attenuation, scattered))
        return attenuation * ray_color(scattered, depth - 1, world);
      return color(0, 0, 0);
    }
    return sky_color(r);
}
\end{lstlisting}

\FloatBarrier
\subsubsection{Intersection test with ray and scene}
This subsection describes how the ray is tested against a set of objects to find the closest intersection, enabling accurate shading calculations. This implementation naively iterates over every object in the scene, without employing more advanced acceleration structures such as Bounding Volume Hierarchies\cite{Kay1986}. See Listing \ref{listing:hittable-list}.

\begin{lstlisting}[caption={\textbf{Hittable List Implementation}}, label={listing:hittable-list}, name=hittable-list, float=htbp, style=mystyle, language=C++]
class hittable_list : public hittable {
  std::vector<shared_ptr<hittable>> objects;

  bool hit(const ray &r, interval ray_t, hit_record &rec) const override {
    hit_record temp_rec;
    bool hit_anything = false;
    auto closest_so_far = ray_t.max;

    for (const auto &object : objects) {
      if (object->hit(r, interval(ray_t.min, closest_so_far), temp_rec)) {
        hit_anything = true;
        closest_so_far = temp_rec.t;
        rec = temp_rec;
      }
    }

    return hit_anything;
  }
};
\end{lstlisting}

\subsection{Serial RT}
\label{subsec:serial-rt}
While no standalone serial implementation was created, the OpenMP implementation with a single-thread parameter (\texttt{OMP\_NUM\_THREADS=1}) serves as the baseline for evaluating parallel performance. This approach processes each pixel sequentially, tracing rays from the camera through the scene, computing intersections, and determining the final color for each pixel. Although straightforward, this single-threaded execution lacks the scalability needed for real-time rendering of complex scenes.

\begin{lstlisting}[caption={\textbf{Serial RT:} The baseline implementation computes the color of each pixel by casting multiple rays per pixel (antialiasing) and aggregating their contributions. Each pixel's color is computed independently.}, label={listing:serial-rt}, name=serial-rt, float=htbp, style=mystyle, language=C++]
for (int y = 0; y < height; y++) {
  for (int x = 0; x < width; x++) {
    color pixel_color = color(0, 0, 0);
    for (int s = 0; s < samples_per_pixel; s++) {
      ray r = get_ray(x, y);
      pixel_color += ray_color(r, max_depth, world);
    }
    image[y * image_width + x] += pixel_color;
  }
}
\end{lstlisting}

\FloatBarrier
\subsection{OpenMP RT}
\label{subsec:openmp-rt}
The parallel implementation leverages OpenMP to distribute the computational workload across multiple threads. By parallelizing the per-pixel ray tracing task, we aim to achieve significant performance improvements over the serial version.

\begin{lstlisting}[caption={\textbf{OpenMP RT:} This parallelized implementation uses OpenMP to distribute pixel computations across multiple threads. The \texttt{\#pragma omp parallel for} directive enables efficient workload distribution, with optional loop collapsing (\texttt{collapse(2)}) for combining nested loops. The scheduling policy is configurable at runtime, allowing flexible tuning to balance performance and overhead.}, label={listing:openmp-rt}, name=openmp-rt, float=htbp, style=mystyle, language=C++]
#if OMP_COLLAPSE
#pragma omp parallel for collapse(2) schedule(runtime)
#else
#pragma omp parallel for schedule(runtime)
#endif
for (int y = 0; y < height; y++) {
  for (int x = 0; x < width; x++) {
    color pixel_color = color(0, 0, 0);
    for (int s = 0; s < samples_per_pixel; s++) {
      ray r = get_ray(x, y);
      pixel_color += ray_color(r, max_depth, world);
    }
    image[y * image_width + x] += pixel_color;
  }
}
\end{lstlisting}

\subsubsection{Parallelization Strategy}
The OpenMP implementation focuses on parallelizing the loop that iterates over image pixels. Each thread processes a subset of the pixels independently, taking advantage of the embarrassingly parallel nature of the problem. Two scheduling strategies were implemented and evaluated:

\begin{itemize}
    \item \textbf{Static Scheduling:} Pixels are evenly divided among threads at compile time. This approach is efficient for simple scenes but can lead to load imbalance in complex ones.
    \item \textbf{Dynamic Scheduling:} Pixels are assigned to threads in smaller chunks at runtime, improving load balancing but introducing overhead due to runtime scheduling decisions.
\end{itemize}



\begin{comment}
%% the material that follows is from the generic tech paper skeleton project

\begin{itemize}
    \item Another way to look at this section is as a paper, within a paper, describing your implementation. That viewpoint makes this the introduction to the subordinate paper, which should describe the overall structure of your implementation and how it is designed to address the problem effectively.
\item Then, describe the structure of the rest of this section, and what each subsection describes.
\end{itemize}

How our solution (will | does) work
\begin{itemize}
    \item This is the body of the subordinate paper describing your solution. It may be divided into several subsections as required by the nature of your implementation.
    \item The level of detail about how the solution works is determined by what is appropriate to the type of paper (conference, journal, technical report).
    \item This section can be fairly short for conference papers, fairly long for journal papers, or quite long in technical reports. It all depends on the purpose of the paper and the target audience.
    \item Proposals are necessarily a good deal more vague in this section since you have to convince someone you know enough to have a good chance of building a solution, but that you have not already done so.
\end{itemize}

\end{comment}
