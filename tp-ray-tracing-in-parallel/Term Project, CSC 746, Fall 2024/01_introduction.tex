\section{Introduction}
\label{sec:introduction}
% For homework writeups, the Introduction section should state the general thrust of the assignment.

% What is the problem being studied? Explain in 2-3 sentences.

% What is the approach for studying the problem? Hint: the approach consists of the program(s) you are writing, so say in 2-3 sentences something about those programs. If you like, it is ok to use a forward reference, and say something like "we present the implementation in \S\ref{sec:implementation}. 

% What are the main results? Say something about the results in 2-3 sentences: what is the nature of your experiment that tests your implementation, and say something about the insights gained. 

% for each of the problem statement, approach statement, and findings/result statement from the abstract, amplify these into a short paragraph for each. Here, "short paragraph" means 3-4 sentences.

Ray tracing is a powerful technique for rendering photorealistic images, but its computational demands make achieving real-time performance a significant challenge. This project focuses on investigating and implementing parallelization strategies to improve ray tracing performance on CPUs, aiming to identify whether ray tracing is "embarrassingly parallel" and how this parallelism can be effectively leveraged for real-time applications.

In this study, we explore the performance gains achieved through the parallel implementation of ray tracing using OpenMP \cite{openmp_spec}. Our results demonstrate that, although ray tracing appears to be "embarrassingly parallel," several factors—such as load imbalance, overhead from workload distribution, and hardware underutilization—can prevent parallelization from achieving optimal speedup.

The rest of this paper first discusses related work in Section \ref{sec:related-work}, providing the context for our study. Section \ref{sec:implementation} describes our implementation, detailing the parallelization strategies used. Section \ref{sec:evaluation} explains the evaluation methodology, presents performance results, and discusses key findings. Section \ref{sec:conclusion-and-future-work} concludes the paper with a summary of our contributions and directions for future work.

\begin{comment}
%% the material that follows is from the generic tech paper skeleton project

The problem we have solved
\begin{itemize}
    \item Concentrate on making this assertion and only this assertion in a succinct set of 1 to 3 paragraphs
    \item A common mistake is to explain too much of the problem context first. Instead, state the problem essentially as a claim, and leave explanations supporting your claim to the next part, “Why it is not already solved.”
\end{itemize}

Why the problem is not already solved or other solutions are ineffective in one or more important ways
\begin{itemize}

\item Your new idea need not solve every problem but it should solve at least one that is not already solved
\item This is the place to provide a succinct description of the problem context giving enough information to support the claim that a problem exists, made in the preceding problem declaration.
  
\end{itemize}

Why our solution is worth considering and why is it effective in some way that others are not

\begin{itemize}
\item A succinct statement of why the reader should care enough to read the rest of the paper.
\item This should include a statement about the characteristics of your solution to the problem which 1) make it a solution, and 2) make it superior to other solutions to the same problem.
\end{itemize}

How the rest of the paper is structured
\begin{itemize}
    \item The short statement below is often all you need, but you should change it when your paper has a different structure, or when more information is required to describe what a given section contains. If it isn’t required then you don’t want to say it here.
\end{itemize}

The rest of this paper first discusses related work in Section 2, and then describes our implementation in Section 3. Section 4 describes how we evaluated our system and presents the results. Section 5 presents our conclusions and describes future work.

\end{comment}