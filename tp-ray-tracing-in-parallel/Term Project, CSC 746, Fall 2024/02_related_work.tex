\section{Related Work}
\label{sec:related-work}
Ray tracing has been a cornerstone of photorealistic rendering since its inception, with significant advancements in both algorithms and hardware optimization.

Phong Shading~\cite{Phong1975} introduced a realistic shading model in 1975 that enhanced 3D rendering by approximating light interactions on surfaces. Whitted's Ray Tracing~\cite{Whitted1980} expanded the paradigm in 1980 to incorporate global illumination effects such as reflections, refractions, and shadows, laying the groundwork for modern physically-based rendering techniques.

Subsequent innovations, such as Distributed Ray Tracing~\cite{Cook1984}, addressed issues like aliasing and simulated effects such as motion blur and soft shadows through distributed sampling. The Rendering Equation~\cite{Kajiya1986} unified various rendering approaches under a single mathematical framework in 1986, enabling sophisticated solutions for complex illumination scenarios.

With the advent of GPUs, significant effort has been devoted to optimizing ray tracing for parallel hardware. For example, GPU Ray Traversal Optimization~\cite{Aila2009} in 2009 identified bottlenecks in memory bandwidth and workload distribution, proposing techniques such as persistent threads to enhance performance.

Despite these advancements, achieving real-time ray tracing remains a challenge, especially on general-purpose hardware. Existing methods often encounter limitations such as load imbalance and inefficient utilization of parallel resources, motivating further exploration of parallelization strategies tailored to CPUs.
