% Please take a few moments and try to compose an abstract for your homework writeup. It should contain these ideas: what was the problem being studied, what was the approach (what did you implement), what are the results.
% The abstract should describe the basic message of the paper, including: the problem, why your solution should be of interest, some notion that your solution is effective, and a teaser about how it has been evaluated. Cover all of this using between 75 and 150 words. Thus, the abstract is the hardest part to write. Sometimes I try to write it first, but the final version is usually composed of items drawn from the introduction, and then condensed, as the last step of writing the paper.

% describes the focus of the study, the approach, and the primary findings/results (3 or 4 sentences total). Writing tip: it's often the case that the Abstract and Introduction are the last items written in a technical paper, once you know the outcome of the performance study.
This study explores the parallelization of ray tracing to determine its feasibility as an "embarrassingly parallel" problem. Using OpenMP\cite{openmp_spec}, we implemented a ray tracing program that runs in parallel on CPUs, analyzing workload distribution and scalability. Our experiments revealed that static scheduling leads to load imbalances, while dynamic scheduling improves scalability at the cost of overhead. The results indicate suboptimal speedup for smaller problem sizes, achieving a speedup of 77.5 times at high scene complexity with 256 threads. This demonstrates that parallelization benefits increase with scene complexity but plateau at higher thresholds. While our findings validate the potential for parallelism in ray tracing, identifying the bottleneck of the current implementation and achieving the ideal speedup on CPUs remain challenges. Evaluation metrics include runtime and speedup, providing insights into workload distribution and its impact on performance scalability.