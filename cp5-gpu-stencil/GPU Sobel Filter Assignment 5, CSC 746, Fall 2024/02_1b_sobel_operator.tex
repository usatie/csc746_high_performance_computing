\subsection{Sobel Operator}
\label{subsec:sobel-operator}

The Sobel operator approximates the gradient of image intensity for edge detection~\cite{kanopoulos1988design}. It uses two \(3 \times 3\) kernels, convolved with the original image \( A \), to compute horizontal and vertical gradients \( G_x \) and \( G_y \):

\[
G_x = \begin{bmatrix}
+1 & 0 & -1 \\
+2 & 0 & -2 \\
+1 & 0 & -1 \\
\end{bmatrix} * A,
\quad
G_y = \begin{bmatrix}
+1 & +2 & +1 \\
0  &  0 &  0 \\
-1 & -2 & -1 \\
\end{bmatrix} * A
\]

The gradient magnitude at each pixel is computed as:

\[
G = \sqrt{G_x^2 + G_y^2}
\]

The function \texttt{sobel\_filtered\_pixel()} computes the Sobel filter at a pixel location, as shown in Listing~\ref{listing:sobel-filtered-pixel}. The arrays \( gx \) and \( gy \) represent the kernel weights for the horizontal and vertical filters, respectively.

\begin{lstlisting}[caption={\textbf{Sobel filtered pixel computation.} Computes the Sobel filter at a specific pixel location.},label={listing:sobel-filtered-pixel},float=htbp,style=mystyle,language=C++]
Function sobel_filtered_pixel(img[width, height], x, y, gx[3, 3], gy[3, 3]) {
    if x or y is at the boundary of the img
        return 0
    Gx = 0.0, Gy = 0.0;
    for j in {0, 1, 2}
        for i in {0, 1, 2}
            xx = x - 1 + i;
            yy = y - 1 + j;
            Gx += img[xx, yy] * gx[i, j];
            Gy += img[xx, yy] * gy[i, j];
    return sqrt(Gx * Gx + Gy * Gy);
}
\end{lstlisting}