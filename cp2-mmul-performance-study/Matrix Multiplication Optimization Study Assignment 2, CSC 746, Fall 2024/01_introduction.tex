\section{Introduction}
\begin{comment}
    for each of the problem statement, approach statement, and findings/result statement from the abstract, amplify these into a short paragraph for each. Here, a short paragraph means 3-4 sentences.
\end{comment}

% For homework writeups, the Introduction section should state the general thrust of the assignment.

% What is the problem being studied? Explain in 2-3 sentences.
% What is the approach for studying the problem? Hint: the approach consists of the program(s) you are writing, so say in 2-3 sentences something about those programs. If you like, it is ok to use a forward reference, and say something like "we present the implementation in \S\ref{sec:implementation}. 
The objective of this assignment is to optimize matrix multiplication by improving cache usage to enhance both spatial and temporal locality. The naive approach to matrix multiplication is inefficient in terms of memory access and locality. To explore how we can optimize the cache usage of matrix multiplication, we implemented three different matrix multiplication approaches: Basic Matrix Multiplication (Basic), Matrix Multiplication using C Basic Linear Algebra Subprograms (CBLAS), and Blocked Matrix Multiplication with Copy Optimization (BMMCO). These implementations were chosen to represent minimal optimization, a state-of-the-art reference, and our own implementation using blocking and copy optimization. 
The performance of these implementations was evaluated on the Perlmutter supercomputer at NERSC, using C++ for the implementations. The key performance metric, MFLOP/s, was calculated based on the elapsed time of each method. Detailed implementations are provided in Section~\ref{sec:implementation}.


% What are the main results? Say something about the results in 2-3 sentences: what is the nature of your experiment that tests your implementation, and say something about the insights gained. 

The results demonstrate that blocking and copy optimization techniques can significantly enhance the computational throughput of matrix multiplication compared to Basic MM by successfully utilizing both spatial and temporal locality. However, despite these gains, there remains a considerable performance gap compared to the CBLAS implementation. This performance disparity suggests that further optimizations are possible, such as loop reordering, improved in-memory data layout strategies, and the incorporation of advanced matrix multiplication algorithms like Strassen’s algorithm.

\begin{comment}
%% the material that follows is from the generic tech paper skeleton project

The problem we have solved
\begin{itemize}
    \item Concentrate on making this assertion and only this assertion in a succinct set of 1 to 3 paragraphs
    \item A common mistake is to explain too much of the problem context first. Instead, state the problem essentially as a claim, and leave explanations supporting your claim to the next part, “Why it is not already solved.”
\end{itemize}

Why the problem is not already solved or other solutions are ineffective in one or more important ways
\begin{itemize}

\item Your new idea need not solve every problem but it should solve at least one that is not already solved
\item This is the place to provide a succinct description of the problem context giving enough information to support the claim that a problem exists, made in the preceding problem declaration.
  
\end{itemize}

Why our solution is worth considering and why is it effective in some way that others are not

\begin{itemize}
\item A succinct statement of why the reader should care enough to read the rest of the paper.
\item This should include a statement about the characteristics of your solution to the problem which 1) make it a solution, and 2) make it superior to other solutions to the same problem.
\end{itemize}

How the rest of the paper is structured
\begin{itemize}
    \item The short statement below is often all you need, but you should change it when your paper has a different structure, or when more information is required to describe what a given section contains. If it isn’t required then you don’t want to say it here.
\end{itemize}

The rest of this paper first discusses related work in Section 2, and then describes our implementation in Section 3. Section 4 describes how we evaluated our system and presents the results. Section 5 presents our conclusions and describes future work.

\end{comment}