% Please take a few moments and try to compose an abstract for your homework writeup. It should contain these ideas: what was the problem being studied, what was the approach (what did you implement), what are the results.
% The abstract should describe the basic message of the paper, including: the problem, why your solution should be of interest, some notion that your solution is effective, and a teaser about how it has been evaluated. Cover all of this using between 75 and 150 words. Thus, the abstract is the hardest part to write. Sometimes I try to write it first, but the final version is usually composed of items drawn from the introduction, and then condensed, as the last step of writing the paper.

% describes the focus of the study, the approach, and the primary findings/results (3 or 4 sentences total). Writing tip: it's often the case that the Abstract and Introduction are the last items written in a technical paper, once you know the outcome of the performance study.

This assignment investigates two performance optimization techniques—parallelism and cache utilization—by evaluating the performance of three matrix multiplication (MM) methods: Basic Matrix Multiplication with OpenMP (Basic OMP), Blocked Matrix Multiply with Copy Optimization (BMMCO OMP), and CBLAS, using hardware counters. The implementations were tested across varying matrix sizes (\(128 \times 128\), \(512 \times 512\), and \(2048 \times 2048\)) and thread counts (1, 4, 16, and 64) to assess speedup and efficiency. Results indicate that while Basic OMP scales effectively with increasing thread count, BMMCO OMP exhibits limited scalability for smaller matrices and larger block sizes due to thread underutilization. The BMMCO OMP method significantly reduced L2 and L3 cache accesses, particularly with larger block sizes, resulting in superior performance compared to Basic OMP. These findings suggest that achieving optimal performance requires balancing between increasing cache utilization through larger block sizes and maximizing thread utilization with smaller block sizes.