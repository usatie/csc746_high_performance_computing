% State the objective for the implementation (2-3 sentences).
The objective of the CBLAS implementation is to establish a baseline for the performance of highly optimized MM, without manual parallelization. This implementation uses the CBLAS library \cite{cblas} to achieve efficient serial MM, serving as a reference to evaluate the benefits of parallelization using OpenMP in the other implementations.

% Describe the implementation (2-3 sentences, or more if needed).
The CBLAS implementation, as shown in Listing~\ref{listing:cblas}, wraps a call to the highly optimized CBLAS routine \textit{cblas\_dgemm}, which performs matrix-MM. We placed the LIKWID markers to capture performance metrics, ensuring that the collected data accurately reflects the computation time for this highly optimized serial method.

\begin{lstlisting}[caption={\textbf{CBLAS implementation using the cblas\_dgemm routine for optimized serial MM.} LIKWID markers are used to measure performance data specific to the CBLAS execution.},label={listing:cblas},name=cblas,float=htbp,style=mystyle,language=C++]
void square_dgemm(int n, double* A, double* B, double* C) {
    LIKWID_MARKER_START(MY_MARKER_REGION_NAME);
    cblas_dgemm(CblasRowMajor, CblasNoTrans, CblasNoTrans, n, n, n, 1., A, n, B, n, 1., C, n);
    LIKWID_MARKER_STOP(MY_MARKER_REGION_NAME);
}
\end{lstlisting}