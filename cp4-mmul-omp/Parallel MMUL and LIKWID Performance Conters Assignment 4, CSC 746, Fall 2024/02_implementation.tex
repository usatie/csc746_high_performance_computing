
\section{Implementation}
\label{sec:implementation}
% Put an introductory paragraph here that gives the reader an overview of what's coming. If there are multiple subsections, say in a few words or a sentence something about each subsection.

% include a separate subsection for each of the different implementations. Briefly describe your implementation, and include the use of compact pseudocode as necessary. The focus here should be on conciseness and clarity. Be sure to describe the strategy you used in parallelizing your code. Your example Listings should clearly indicate the OpenMP pragmas you used. 

This section presents three MM implementations aimed at evaluating performance improvements through parallelization and optimization techniques. The first implementation, Basic OMP, evaluates the impact of adding shared-memory parallelism to a simple MM approach. The second, Blocked Matrix Multiplication with Copy Optimization (BMMCO), combines cache optimization techniques with parallelization to further enhance computational efficiency. Finally, the CBLAS MM serves as a baseline, using a highly optimized but serial implementation.

Each subsection details the objectives, describes the applied techniques, and presents compact pseudocode or listings to illustrate the use of OpenMP and optimization strategies.

\subsection{Basic OMP}
\label{subsec:basic-omp}
% State the objective for the implementation (2-3 sentences).
The objective of the Basic OMP implementation is to evaluate the effectiveness of parallelizing a MM operation using OpenMP. By implementing parallelization with OpenMP, this version aims to demonstrate performance improvements over the basic serial implementation, serving as a benchmark to understand scaling characteristics in shared-memory parallel environments.

% Describe the implementation (2-3 sentences, or more if needed).
The Basic OMP implementation employs a triply-nested loop to perform MM, where OpenMP is used to parallelize the outermost loop. This allows each iteration of the loop, representing a row computation, to be handled by a different thread. We placed LIKWID markers before and after the main MM operations, ensuring that the collected performance data accurately reflects the parallelized portion of the computation. The source code for this implementation is presented in Listing~\ref{listing:basic-omp}.


\begin{lstlisting}[caption={\textbf{Basic OpenMP implementation of MM.} The implementation uses a triply-nested loop with OpenMP parallelism. LIKWID markers are added within the parallel block, specifically before and after the core matrix multiply code, to ensure performance data collection is focused only on the parallel MM operations.},label={listing:basic-omp},name=basic-serial,float=htbp,style=mystyle,language=C++]
void square_dgemm(int n, double* A, double* B, double* C) 
{
    #pragma omp parallel for
    for (int i = 0; i < n; ++i) {        
        LIKWID_MARKER_START(MY_MARKER_REGION_NAME);

        for (int j = 0; j < n; ++j) {
            double cij = C[i*n+j];
            for (int k = 0; k < n; ++k) {
                cij += A[i*n+k] * B[k*n+j];
            }
            C[i*n+j] = cij;
        }
        LIKWID_MARKER_STOP(MY_MARKER_REGION_NAME);
    }
}
\end{lstlisting}


\FloatBarrier
\subsection{CBLAS}
\label{subsec:cblas}
% State the objective for the implementation (2-3 sentences).
The objective of the CBLAS implementation is to establish a baseline for the performance of highly optimized MM, without manual parallelization. This implementation uses the CBLAS library \cite{cblas} to achieve efficient serial MM, serving as a reference to evaluate the benefits of parallelization using OpenMP in the other implementations.

% Describe the implementation (2-3 sentences, or more if needed).
The CBLAS implementation, as shown in Listing~\ref{listing:cblas}, wraps a call to the highly optimized CBLAS routine \textit{cblas\_dgemm}, which performs matrix-MM. We placed the LIKWID markers to capture performance metrics, ensuring that the collected data accurately reflects the computation time for this highly optimized serial method.

\begin{lstlisting}[caption={\textbf{CBLAS implementation using the cblas\_dgemm routine for optimized serial MM.} LIKWID markers are used to measure performance data specific to the CBLAS execution.},label={listing:cblas},name=cblas,float=htbp,style=mystyle,language=C++]
void square_dgemm(int n, double* A, double* B, double* C) {
    LIKWID_MARKER_START(MY_MARKER_REGION_NAME);
    cblas_dgemm(CblasRowMajor, CblasNoTrans, CblasNoTrans, n, n, n, 1., A, n, B, n, 1., C, n);
    LIKWID_MARKER_STOP(MY_MARKER_REGION_NAME);
}
\end{lstlisting}

\FloatBarrier
\subsection{BMMCO OMP}
\label{subsec:bmmco-omp}
% State the objective for the implementation (2-3 sentences).
The objective of the BMMCO with OpenMP implementation is to evaluate how effectively parallelization, combined with cache optimization techniques like blocking and copy optimization, can enhance MM performance. This approach aims to reduce cache misses and improve data locality, leveraging these optimizations alongside OpenMP to achieve superior efficiency.

% Describe the implementation (2-3 sentences, or more if needed).
The BMMCO OMP implementation divides matrices into smaller blocks, enabling operations on submatrices to improve data locality and make better use of cache memory. Copying these blocks into local storage allows faster cache access, reducing repeated memory costs. OpenMP parallelizes the outermost loop, allowing each block row of C to be processed concurrently by different threads, which enhances computational efficiency and reduces overall runtime. The implementation is shown in Listing~\ref{listing:BMMCO-omp}.

\begin{lstlisting}[caption={\textbf{BMMCO with OpenMP implementation of MM using blocking and copy optimization techniques.} The implementation divides the matrices into smaller sub-blocks to improve data locality and minimize cache misses. OpenMP is used to parallelize the outermost loop, allowing each block row to be processed concurrently by different threads. Copying matrix blocks into local storage (faster cache) prior to performing computations helps reduce memory latency by taking advantage of the CPU cache, rather than accessing global memory repeatedly. LIKWID markers are included to measure the performance of the parallelized block operations, focusing specifically on the parallel region of the MM.},label={listing:BMMCO-omp}, name=BMMCO-omp, float=htbp, style=mystyle, language=C++]
void square_dgemm_blocked(int n, int block_size, double* A, double* B, double* C)
{
#define MAX_BLOCK (64 * 64)
    static double aBlock[MAX_BLOCK], bBlock[MAX_BLOCK], cBlock[MAX_BLOCK];
    int Nb = n / block_size;
    #pragma omp parallel for private(aBlock, bBlock, cBlock)
    for (int i = 0; i < Nb; i++) {
        LIKWID_MARKER_START(MY_MARKER_REGION_NAME);

        for (int j = 0; j < Nb; j++) {
            blockread(cBlock, C, i, j, n, block_size);
            for (int k = 0; k < Nb; k++) {
                blockread(aBlock, A, i, k, n, block_size);
                blockread(bBlock, B, k, j, n, block_size);
                // Perform matrix multiplication on the copied blocks
                square_dgemm(block_size, aBlock, bBlock, cBlock);
            }
            blockwrite(cBlock, C, i, j, n, block_size);
        }
        LIKWID_MARKER_STOP(MY_MARKER_REGION_NAME);
    }
}
\end{lstlisting}


\begin{comment}
%% the material that follows is from the generic tech paper skeleton project

\begin{itemize}
    \item Another way to look at this section is as a paper, within a paper, describing your implementation. That viewpoint makes this the introduction to the subordinate paper, which should describe the overall structure of your implementation and how it is designed to address the problem effectively.
\item Then, describe the structure of the rest of this section, and what each subsection describes.
\end{itemize}

How our solution (will | does) work
\begin{itemize}
    \item This is the body of the subordinate paper describing your solution. It may be divided into several subsections as required by the nature of your implementation.
    \item The level of detail about how the solution works is determined by what is appropriate to the type of paper (conference, journal, technical report).
    \item This section can be fairly short for conference papers, fairly long for journal papers, or quite long in technical reports. It all depends on the purpose of the paper and the target audience.
    \item Proposals are necessarily a good deal more vague in this section since you have to convince someone you know enough to have a good chance of building a solution, but that you have not already done so.
\end{itemize}

\end{comment}
